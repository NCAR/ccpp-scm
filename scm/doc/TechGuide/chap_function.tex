\chapter{Algorithm}
\label{chapter: algorithm}

\section{Algorithm Overview}

Like most SCMs, the algorithm for the CCPP SCM is quite simple. In a nutshell, the SCM code performs the following:
\begin{itemize}
\item Read in an initial profile and the forcing data.
\item Create a vertical grid and interpolate the initial profile and forcing data to it.
\item Initialize the physics suite.
\item Perform the time integration, applying forcing and calling the physics suite each time step.
\item Output the state and physics data.
\end{itemize}
In this chapter, it will briefly be described how each of these tasks is performed.

\section{Reading input}
The following steps are performed at the beginning of program execution:
\begin{enumerate}
\item Call \execout{get\_config\_nml()} in the \execout{scm\_input} module to read in the \hyperref[subsection: case config]{\execout{case\_config}} and \hyperref[subsection: physics config]{\execout{physics\_config}} namelists. This subroutine also sets some variables within the \execout{scm\_state} derived type from the data that was read.
\item Call \execout{get\_case\_init()} (or \execout{get\_case\_init\_DEPHY()} if using the DEPHY format) in the \execout{scm\_input} module to read in the \hyperref[subsection: case input]{case input data file}. This subroutine also sets some variables within the \execout{scm\_input} derived type from the data that was read.
\item Call \execout{get\_reference\_profile()} in the \execout{scm\_input} module to read in the reference profile data. This subroutine also sets some variables within the \execout{scm\_reference} derived type from the data that was read. At this time, there is no ``standard'' format for the reference profile data file. There is a \execout{select case} statement within the \execout{get\_reference\_profile()} subroutine that reads in differently-formatted data. If adding a new reference profile, it will be required to add a section that reads its data in this subroutine.
\end{enumerate}

\section{Setting up vertical grid and interpolating input data}
The CCPP SCM uses pressure for the vertical coordinate (lowest index is the surface). The pressure levels are calculated using the surface pressure and coefficients ($a_k$ and $b_k$), which are taken directly from FV3 code (\execout{fv\_eta.h}). For vertical grid options, inspect \execout{scm/src/scm\_vgrid.F90} for valid values of \execout{npz\_type}. The default vertical coordinate uses 127 levels and sets \execout{npz\_type} to the empty string. Alternatively, one can specify the ($a_k$ and $b_k$) coefficients via an external file in the \execout{scm/data/vert\_coord\_data} directory and pass it in to the SCM via the \execout{-{}-vert\_coord\_file} argument of the run script. If changing the number of vertical levels or the algorithm via the \execout{-{}-levels}  or \execout{-{}-npz\_type} run script arguments, be sure to check \execout{src/scm/scm\_vgrid.F90} and \execout{fv\_eta.h} that the vertical coordinate is as inteneded.

After the vertical grid has been set up, the state variable profiles stored in the \execout{scm\_state} derived data type are interpolated from the input and reference profiles in the \execout{set\_state} subroutine of the \execout{scm\_setup} module.

\section{Physics suite initialization}
\label{section: physics init}
With the CCPP framework, initializing a physics suite is a 3-step process:
\begin{enumerate}
\item Initialize variables needed for the suite initialization routine. For suites originating from the GFS model, this involves setting some values in a derived data type used in the initialization subroutine. Call the suite initialization subroutine to perform suite initialization tasks that are not already performed in the \execout{init} routines of the CCPP-compliant schemes (or associated initialization stages for groups or suites listed in the suite definition file). Note: As of this release, this step will require another suite intialization subroutine to be coded for a non-GFS-based suite to handle any initialization that is not already performed within CCPP-compliant scheme initialization routines.
\item Associate the \execout{scm\_state} variables with the appropriate pointers in the \execout{physics} derived data type.
\item Call \execout{ccpp\_physics\_init} with the \execout{cdata} derived data type as input. This call executes the initialization stages of all schemes, groups, and suites that are defined in the suite definition file.
\end{enumerate}

\section{Time integration}
\label{section: time integration}
The CCPP SCM uses the simple forward Euler scheme for time-stepping.

During each step of the time integration, the following sequence occurs:
\begin{enumerate}
\item Update the elapsed model time.
\item Calculate the current date and time given the initial date and time and the elapsed time.
\item Call the \execout{interpolate\_forcing()} subroutine in the \execout{scm\_forcing} module to interpolate the forcing data in space and time.
\item Recalculate the pressure variables (pressure, Exner function, geopotential) in case the surface pressure has changed.
\item Call \execout{do\_time\_step()} in the \execout{scm\_time\_integration} module. Within this subroutine:
\begin{itemize}
\item Call the appropriate \execout{apply\_forcing\_*} subroutine from the \execout{scm\_forcing} module.
\item For each column, call \execout{ccpp\_physics\_run()} to call all physics schemes within the suite (this assumes that all suite parts are called sequentially without intervening code execution)
\end{itemize}
\item Check to see if output should be written during the current time step and call \execout{output\_append()} in the \execout{scm\_output} module if necessary.
\end{enumerate}

\section{Writing output}
Output is accomplished via writing to a NetCDF file. If not in the initial spin-up period, a subroutine is called to determine whether data needs to be added to the output file during every timestep. Variables can be written out as instantaneous or time-averaged and there are 5 output periods:
\begin{enumerate}
\item one associated with how often instantaneous variables should be written out (controlled by the \execout{-{}- n\_itt\_out} run script variable).
\item one associated with how often diagnostic (either instantaneous or time-averaged) should be written out (controlled by the \execout{-{}- n\_itt\_diag} run script variable)
\item one associated with the shortwave radiation period (controlled by \execout{fhswr} variable in the physics namelist)
\item one associated with the longwave radiation period (controlled by the \execout{fhlwr} variable in the physics namelist)
\item one associated with the minimum of the shortwave and longwave radiation intervals (for writing output if any radiation is called)
\end{enumerate}

Further, which variables are output and on each interval are controlled via the \execout{scm/src/scm\_output.F90} source file. Of course, any changes to this file must result in a recompilation to take effect. There are several subroutines for initializing the output file (\execout{output\_init\_*}) and for appending to it (\execout{output\_append\_*}) that are organized according to their membership in physics derived data types. See the \execout{scm/src/scm\_output.F90} source file to understand how to change output variables.
